

\chapter*{Abstract} 								% Keine Nummerierung

	\addcontentsline{toc}{chapter}{Abstract}	% Damit es im Inhaltsverzeichnis erscheint

In electromagnetics, inverse source solvers are used for solving inverse scattering problems for example by computing equivalent current distributions from a set of measurement samples, usually taken in the near-field of the device under test (DUT). Using the equivalent sources, one can perform important field transformations, which are beneficial in various high-frequency engineering problems, such as antenna measurement and electromagnetic compatibility. It is known that numerical solution methods for some of the inverse scattering problems based on integral equations suffer from a so-called low-frequency breakdown. This means, below a certain frequency the solver becomes inaccurate. The objective of this work is to design and implement various test circuits with diverse near-field patterns in Computer Simulation Technology Studio Suite (CST) to test an inverse source solver at low frequencies. In this context, this report first elaborates the methods for 3D modeling of various SMD components in CST, specifically, drum core inductors with a ferrite core, air core inductors, multilayer ceramic capacitors, and chip resistors. Building on the component models, the 3D models of the test circuits, such as various LC bandpass filters, Wilkinson dividers and hybrid couplers are implemented in CST, out of which two bandpass filters and one Wilkinson divider is examined in detail in this report. Additionally, other basic test circuits are implemented and examined including the loop circuits with a dominant magnetic field at low frequencies, microstrip circuits with a dominant electric field at low frequencies, and a circuit that contains both electric and magnetic fields at low frequencies. Finally, the circuits are simulated and the results in the near-field are shown.


%\chapter*{Kurzbeschreibung} 							% Keine Nummerierung
%
%	\addcontentsline{toc}{chapter}{Kurzbeschreibung}	% Damit es im Inhaltsverzeichnis erscheint
%	
%Deutsche Beschreibung. 

