
%!TEX program = lualatex

\documentclass[a4paper,fleqn, headsepline, parskip=half, pagesize=dvips, twoside, final, openright, numbers=noenddot]{scrbook}

\usepackage{../../hft-thesis} % use same style as main document
\usepackage{../MyFigureStyle} % supplements for images


  

\begin{document}

\thispagestyle{empty}

\usepgfplotslibrary{colormaps}
\pgfplotsset{colormap/hot2}


\newcommand*{\arrowheadthreeD}[4]{%
	\colorlet{beamcolor}{#1!75!black}
	\colorlet{innercolor}{#1!50}
	\foreach \i in {1, 0.9, ..., 0} { %% the step 1,0.9 defines the number of lines for the head. For bigger files with smoother transitions change to e.g. 0.99 
		\pgfmathsetmacro{\shade}{\i*\i*100}
		\pgfmathsetmacro{\startangle}{90-\i*30}
		\pgfmathsetmacro{\endangle}{90+\i*30}
		\fill[beamcolor!\shade!innercolor,shift={#2},rotate=#3,line width=0,line cap=butt,]%,
		(0,0) -- (\startangle:0.259) arc (\startangle:\endangle:0.259)--cycle;
	}
	\fill[beamcolor,shift={#2},rotate=#3,line width=0,line cap=butt] (60:0.26) arc (60:120:0.26) -- (120:0.25) arc (120:60:0.25) -- cycle; 
}

\newcommand*{\arrowthreeD}[4]{
	\begin{scope}[scale=#3,]
		\fill [left color=#1!75!black,right color=#1!75!black,middle color=#1!50,join=round,line cap=round,line width=0,draw=none,shading angle=#4+90,,shift={(0,0.25)},rotate=180] (0,0.25) -- (0.05,0.25) -- (0.05,0.175+0.25) arc (0:180:0.05 and 0.05) -- (-0.05,0.25)--cycle;	  
		\arrowheadthreeD{#1}{(0,0.25)}{180}{#3};
	\end{scope}
} 

\makeatletter
\def\pgfplotsplothandlerquiver@vis@path#1{%
	#1%
	\pgfmathsetmacro\pgfplots@quiver@x{\pgf@x}\global\let\pgfplots@quiver@x\pgfplots@quiver@x%
	\pgfmathsetmacro\pgfplots@quiver@y{\pgf@y}\global\let\pgfplots@quiver@y\pgfplots@quiver@y%
	\pgfplotsaxisvisphasetransformcoordinate\pgfplots@quiver@u\pgfplots@quiver@v\pgfplots@quiver@w%
	\pgfplotsqpointxy{\pgfplots@quiver@u}{\pgfplots@quiver@v}%
	\pgfmathsetmacro\pgfplots@quiver@u{\pgf@x-\pgfplots@quiver@x}%
	\pgfmathsetmacro\pgfplots@quiver@v{\pgf@y-\pgfplots@quiver@y}%
	\pgfmathparse{atan2(\pgfplots@quiver@v,\pgfplots@quiver@u)-90}
	\pgfmathsetmacro\pgfplots@quiver@a{\pgfmathresult}\global\let\pgfplots@quiver@a\pgfplots@quiver@a%
	{%
		\pgftransformshift{\pgfpoint{\pgfplots@quiver@x}{\pgfplots@quiver@y}}%
		\pgfpathmoveto{\pgfpointorigin}%
		\pgfpathlineto{\pgfpoint\pgfplots@quiver@u\pgfplots@quiver@v}%
	}%
}%



%%%%%%%%%%%%%%%%%%%%%%%%%%%%%%%%%%%%%%
%%%%%%%%%%%%%%%%%%%%%%%%%%%%%%%%%%%%%%
%%%%%%%%%%%%%%%%%%%%%%%%%%%%%%%%%%%%%%
% Quiver plot (from data file)
%%%%%%%%%%%%%%%%%%%%%%%%%%%%%%%%%%%%%%
%%%%%%%%%%%%%%%%%%%%%%%%%%%%%%%%%%%%%%
%%%%%%%%%%%%%%%%%%%%%%%%%%%%%%%%%%%%%%


\begin{tikzpicture}
\begin{axis}[
width= 0.51\linewidth,
scale only axis,
legend style={font=\footnotesize },
ymin=2, 
ymax=6,
xmin=3,
xmax=5,
axis equal image,
clip=false,
minor y tick num={1},
minor x tick num={1},
yticklabels={},
xticklabels={},
grid=none,
colormap={cropped hot2}{
	indices of colormap={
		0,...,6.5 of hot2}
},
]

% MSH
\draw[gray,join=round,line cap=round,]  (5,4) -- (4,2) -- (3,4) ;
\draw[gray,join=round,line cap=round,]          (5,4) -- (4,6) -- (3,4) ;

\draw[gray,join=round,line cap=round,] (3,4) -- (5,4);



\addplot[join=round,line cap=round,very thin,
point meta={\thisrow{meta}},
point meta min=0,
quiver={u=\thisrow{u},
	v=\thisrow{v},
	every arrow/.append style={%
		draw=none,
	},
	after arrow/.code={
		\relax{
			\pgftransformshift{\pgfpoint{\pgfplots@quiver@x}{\pgfplots@quiver@y}}%
			\pgftransformrotate{\pgfplots@quiver@a}%
			\arrowthreeD{mapped color}{0,0}{sqrt{\pgfplotspointmetatransformed}/30}{\pgfplots@quiver@a}
		}
	}
},
] table {rwg_quiver.txt};
\end{axis}
\end{tikzpicture} 



\end{document}


